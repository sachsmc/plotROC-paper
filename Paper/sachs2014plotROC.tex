\documentclass[article]{jss}

\usepackage[utf8]{inputenc}
\usepackage{tikz}
\usepackage{natbib}

%%%%%%%%%%%%%%%%%%%%%%%%%%%%%%
%% declarations for jss.cls %%%%%%%%%%%%%%%%%%%%%%%%%%%%%%%%%%%%%%%%%%
%%%%%%%%%%%%%%%%%%%%%%%%%%%%%%

%% almost as usual
\author{		Michael C. Sachs\\Biometric Research Branch, Division of Cancer Treatment and Diagnosis,
National Cancer Institute		}

\title{\pkg{plotROC}: An \proglang{R} Package to Improve the State of ROC Curve
Plots in the Medical Literature}

%% for pretty printing and a nice hypersummary also set:
\Plainauthor{ Michael C. Sachs} %% comma-separated
\Plaintitle{plotROC: An R Package to Improve the State of ROC Curve Plots in the
Medical Literature} %% without formatting

\Shorttitle{\pkg{plotROC}: An \proglang{R} Package for ROC Curve Plots} %% a short title (if necessary)
%% an abstract and keywords
\Abstract{
  Plots of the receiver operating characteristic (ROC) curve are
  ubiquitous in medical research. Designed to simultaneously display the
  operating characteristics at every possible value of a continuous
  diagnostic test, ROC curves are used in oncology to evaluate screening,
  diagnostic, prognostic and predictive biomarkers. We evaluate current
  trends in the design of ROC curve plots with a literature review. Our
  review suggests that ROC curve plots are often ineffective as
  statistical charts and that poor design obscures the relevant
  information the chart is intended to display. We describe our new
  \proglang{R} package that was created to address the shortcomings of
  existing tools. The package has functions to create informative ROC
  curve plots, with sensible defaults, for use in print or as an
  interactive web-based plot. A web application was developed to reach a
  broader audience of scientists who do not use R.
}
\Keywords{ROC curves; graphics; interactive; plots}
\Plainkeywords{ROC curves; graphics; interactive; plots} %% without formatting
%% at least one keyword must be supplied

%% publication information
%% NOTE: Typically, this can be left commented and will be filled out by the technical editor
%% \Volume{50}
%% \Issue{9}
%% \Month{June}
%% \Year{2012}
%% \Submitdate{2012-06-04}
%% \Acceptdate{2012-06-04}

%% The address of (at least) one author should be given
%% in the following format:
\Address{
    Michael C. Sachs\\
    9609 Medical Center Drive, MSC 9735\\
    Bethesda, MD 20892\\
    telephone: 240-276-7888  \\\email{michael.sachs@nih.gov}
}
%% It is also possible to add a telephone and fax number
%% before the e-mail in the following format:
%% Telephone: +43/512/507-7103
%% Fax: +43/512/507-2851

%% for those who use Sweave please include the following line (with % symbols):
%% need no \usepackage{Sweave.sty}

%% end of declarations %%%%%%%%%%%%%%%%%%%%%%%%%%%%%%%%%%%%%%%%%%%%%%%


\begin{document}

\section{Introduction}\label{introduction}

\subsection{About ROC Curves}\label{about-roc-curves}

The Receiver Operating Characteristic (ROC) curve is used to assess the
accuracy of a continuous measurement for predicting a binary outcome. In
medicine, ROC curves have a long history of use for evaluating
diagnostic tests in radiology and general diagnostics. ROC curves have
also been used for a long time in signal detection theory.

The accuracy of a diagnostic test can be evaluated by considering the
two possible types of errors: false positives, and false negatives. For
a continuous measurement that we denote as \(M\), convention dictates
that a test positive is defined as \(M\) exceeding some fixed threshold
\(c\): \(M > c\). In reference to the binary outcome that we denote as
\(D\), a good outcome of the test is when the test is positive among an
individual who truly has a disease: \(D = 1\). A bad outcome is when the
test is positive among an individual who does not have the disease
\(D = 0\).

Formally, for a fixed cutoff \(c\), the true positive fractions is the
probability of a test positive among the diseased population:

\[ TPF(c) = P\{ M > c | D = 1 \} \]

and the false positive fraction is the probability of a test positive
among the healthy population:

\[ FPF(c) = P\{ M > c | D = 0 \} \]

Since the cutoff \(c\) is not usually fixed in advance, we can plot the
TPF against the FPF for all possible values of \(c\). This is exactly
what the ROC curve is, \(FPF(c)\) on the \(x\) axis and \(TPF(c)\) along
the \(y\) axis. A useless test that is not informative at all in regards
to the disease status has \(TPF(c) = FPF(c)\) for all \(c\). The ROC
plot of a useless test is thus the diagonal line. A perfect test that is
completely informative about disease status has \(TPF(c) = 1\) and
\(FPF(c) = 0\) for all \(c\).

Given a sample of test and disease status pairs,
\((M_1, D_1), \ldots, (M_n, D_n)\), we can estimate the ROC curve by
computing proportions in the diseased and healthy subgroups separately.
Specifically, given a fixed cutoff \(c\), an estimate of the \(TPF(c)\)
is

\[ \widehat{TPF(c)} = \frac{\sum_{i = 1}^n 1\{M_i > c\} \cdot 1\{D_i = 1\}}{\sum_{i=1}^n 1\{D_i = 1\}}, \]

where \(1\{\cdot\}\) is the indicator function. An estimate for
\(FPF(c)\) is given by a similar expression with \(D_i = 1\) replaced
with \(D_i = 0\). Calculating these proportions for \(c\) equal to each
unique value of the observed \(M_i\) yields what is known as the
empirical ROC curve estimate. The empirical estimate is a step function.
Other methods exist to estimate the ROC curve, such as the binormal
parametric estimate which can be used to get a smooth curve. There are
also extensions that allow for estimation with time-to-event outcomes
subject to censoring. For a more thorough reference on the methods and
theory surrounding ROC curves, we refer interested readers to
\citet{pepe2003statistical}.

A common way to summarize the value of a test for classifying disease
status is to calculate the area under the ROC curve (AUC). The greater
the AUC, the more informative the test. The AUC summarizes the
complexities of the ROC curve into a single number and therefore is
widely used to facilitate comparisons between tests and populations. It
has been criticized for the same reason because it does not fully
characterize the trade-offs between false- and true-positives.

\subsection{Design Considerations}\label{design-considerations}

The main purpose of visually displaying the ROC curve is to show the
trade-off between the FPF and TPF as the cutoff \(c\) varies. This can
be useful for aiding viewers in choosing an optimal cutoff for decision
making, for comparing a small number of candidate tests, and for
generally illustrating the performance of the test as a classifier. In
practice, once the FPF and TPF are computed for each unique observed
cutoff value, they can be plotted as a simple line chart or scatter plot
using standard statistical plotting tools. This often leads to the
unfortunate design choice of obscuring the critical and useful third
dimension, the range of cutoff values \(c\).

Another key design element is the use of a diagonal guideline for
comparison. The diagonal guideline serves as a baseline for comparison,
allowing observers to roughly estimate the area between the diagonal and
the estimated ROC curve, which serves as a proxy for estimating the
value of the test for classification above a coin-flip. Likewise,
gridlines inside the plotting region and carefully selected axes allow
for accurate assessment of the TPF and FPF at particular points along
the ROC curve. Many medical studies use ROC curves to compare a
multitude of candidate tests to each other and to the null diagonal. In
those cases, curves need to be distinguished by using a legend combined
with different colors or line types, or direct labels inside the
plotting region.

In the medical literature, FPF and TPF are usually referred to in terms
of the jargon terms Sensitivity and Specificity. Sensitivity is
equivalent to the true positive fraction. Specificity is 1 - FPF, the
true negative fraction. Sometimes, the FPF and TPF are incorrectly
referred to as rates, using the abbreviations FPR and TPR. These are
probabilities and their estimates are proportions, therefore we prefer
the use of the term fraction as opposed to rate.

\subsection{Existing Plotting
Software}\label{existing-plotting-software}

The ROC curve plot is, at the most basic level, a line graph. Therefore,
once the appropriate statistics are estimated, existing plotting
functions can be used to create an ROC curve plot. The addition of axis
labels, titles, legends, and so on, indicate a chart as such. In our
literature review, we observed plots with the distinctive
characteristics of the base plotting functions from Microsoft Office,
\proglang{SAS}, \proglang{SPSS}, and the base \proglang{R} plotting
functions.

There are several \proglang{R} packages related to ROC curve estimation
that contain dedicated plotting functions. The \pkg{ROCR} package
\citep{rocr} plots the FPF versus TPF, as usual, and then takes the
interesting approach of encoding the cutoff values as a separate color
scale along the ROC curve itself. A legend for the color scale is place
along the vertical axis on the right of the plotting region. The
\pkg{pROC} package \citep{pROC} is mainly focused on estimating
confidence intervals and regions for restricted ranges of the ROC curve.
The plotting methods therein use the base \proglang{R} plotting
functions to create nice displays of the curves along with shaded
confidence regions.

\subsection{Motivation}\label{motivation}

Anyone giving a cursory look at any of the major medical journals is
likely to find at least one ROC curve plot. We sought assess the usage
of ROC curve plots and to evaluate the design choices made in the
current oncology literature by conducting a small literature review. We
searched Pubmed for clinical trials or observational studies in humans
reported in major oncology journals for the past 10 years for the terms
``ROC Curve'' OR ``ROC Analysis'' OR ``Receiver operating characteristic
curve''. The search was conducted on October 8, 2014 and returned 54
papers. From those papers, 47 images were extracted and reviewed. The
exact specifications for the Pubmed query are available in the
supplementary materials.

Each image consisted of a single ROC curve plot or a panel of multiple
plots. Each plot was inspected manually for the following design
features: the number of curves displayed, the type of axis labels
(sensitivity/ 1 - specificity or true/false positive fractions),
presence or absence of grid lines, presence or absence of diagonal guide
line, whether any cutpoint were indicated, the type of curve label
(legend or direct label), and presence of other textual annotations such
as the area under the curve. The numerical results of the survey are
summarized in table \ref{table1}.

\begin{table}[ht]
\centering
\begin{tabular}{ll}
  \hline
 & percent (count) \\ 
  \hline
Number of curves &  \\ 
  $\quad$1 & 19.6 (9) \\ 
  $\quad$2 & 43.5 (20) \\ 
  $\quad$3 & 10.9 (5) \\ 
  $\quad$4+ & 26.1 (12) \\ 
  $\quad$Average (SD) & 2.6 (1.5) \\ 
  Axis labels &  \\ 
  $\quad$FPF/TPF & 13.0 (6) \\ 
  $\quad$mixed & 2.2 (1) \\ 
  $\quad$none & 2.2 (1) \\ 
  $\quad$sens/spec & 82.6 (38) \\ 
  Diagonal Guide & 43.5 (20) \\ 
  Gridlines & 17.4 (8) \\ 
  Cutoffs indicated & 15.2 (7) \\ 
  AUC indicated & 50.0 (23) \\ 
  Curve Labels &  \\ 
  $\quad$direct & 10.9 (5) \\ 
  $\quad$legend & 63.0 (29) \\ 
  $\quad$none & 19.6 (9) \\ 
  $\quad$title & 6.5 (3) \\ 
   \hline
\end{tabular}
\caption{Results of a literature review of major oncology journals for ROC curve plots. The rows indicate the frequency and count of key design elements of an ROC curve plot. FPR = False positive rate; TPR = True positive rate; sens = Sensitivity; spec = Specificity; AUC = Area under the Curve} 
\label{table1}
\end{table}

The small minority of the figures make any attempt to indicate the
values of the test cutoffs, which is an integral component of the ROC
curve. We conjecture that this is mainly due to the use of default
plotting procedures in statistical software. The software, by default,
treats the ROC curve as a 2 dimensional object, obscuring the cutoff
dimension. Gridlines and direct labels are also somewhat out of the
ordinary. The absence of these features make accurate determination and
comparison of the values more difficult. Many of the plots included
large tables containing estimates and inference for AUCs, while the ROC
curves themselves, numerous and without clear labels or reference lines,
merely served as decoration. We aim to solve some of these problems by
providing an easy-to-use plotting interface for the ROC curve that
provides sensible defaults.

The panels of figure \ref{figure1} illustrate the most common styles of
ROC curve plots, and the associated design elements. We favor the use of
gridline and reference lines to facilitate accurate readings off of the
axes. Direct labels are preferred over legends because they omit the
additional cognitive step of matching line types to labels. Our
\pkg{plotROC} package additionally provides plotting of cutoff values,
using hover events in interactive use, and direct labels for print use.
Exact confidence regions for points on the ROC curve are optionally
calculated and displayed. Additionally, we use axis scales are adjusted
to be denser near the margins 0 and 1. In medical applications, it is
often necessary to have a very low FPR (less than 10\%, for instance),
therefore the smaller scales are useful for accurately determining
values near the margins. The next section details the usage of the
\pkg{plotROC} \proglang{R} package and these features.

\begin{figure}[htbp]
\centering
\includegraphics{figure/figure1-1.pdf}
\caption{Illustration of design choices in plotting ROC curves. Panel A
shows a sparse ROC curve, with no design additions inside the plotting
region. The plot results in more white space than anything else. It is
difficult to accurately determine values without reference lines. Panel
B shows a plot comparing 2 curves, with different line types and a
legend. AUCs are also given in the legend. Panels B and C add gridlines,
diagonal reference lines, and direct labels. \label{figure1}}
\end{figure}

\section{Usage of the Package}\label{usage-of-the-package}

\subsection{Shiny application}\label{shiny-application}

We created a \pkg{shiny} application \citep{shiny} in order to make the
features more accessible to non-\proglang{R} users. A limited subset of
the functions of the \pkg{plotROC} can be performed on an example
dataset or on data that users upload to the website. Resulting plots can
be saved to the users' machine as a pdf or as a stand-alone html file.
It can be used in any modern web browser with no other dependencies at
the website here: http://sachsmc.shinyapps.io/plotROC.

\subsection{Quick start}\label{quick-start}

After installing, the interactive Shiny application can be run locally.

\begin{verbatim}
shiny_plotROC()
\end{verbatim}

\subsection{Command line basic usage}\label{command-line-basic-usage}

We start by creating an example data set. The marker we generate is
moderately accurate for predicting disease status.

\begin{verbatim}
library(plotROC)
D.ex <- rbinom(100, size = 1, prob = .5)
M.ex <- rnorm(100, mean = D.ex)
\end{verbatim}

Next we use the \texttt{calculate\_roc} function to compute the
empirical ROC curve. The disease status need not be coded as 0/1, but if
it is not, \texttt{plotROC} assumes (with a warning) that the lowest
value in sort order signifies disease-free status. This returns a
dataframe with three columns: the cutoff values, the TPF and the FPF.

\begin{verbatim}
rocdata <- calculate_roc(M.ex, D.ex)
str(rocdata)
\end{verbatim}

\begin{verbatim}
'data.frame':   100 obs. of  3 variables:
 $ c  : num  -2.18 -1.67 -1.57 -1.55 -1.54 ...
 $ TPF: num  1 1 1 1 1 1 1 1 1 1 ...
 $ FPF: num  0.981 0.962 0.943 0.925 0.906 ...
\end{verbatim}

The same data.frame \texttt{rocdata} can be used to generate an
interactive plot to view in Rstudio viewer or web browser.

\begin{verbatim}
plot_interactive_roc(rocdata)
\end{verbatim}

The \texttt{rocdata} is passed to the \texttt{ggroc} function with an
optional label. This creates a ggplot object of the ROC curve using the
\pkg{ggplot2} package \citep{ggplot2}.

\begin{verbatim}
myrocplot <- ggroc(rocdata, label = "Example")
\end{verbatim}

We can create an interactive ROC plot using the
\texttt{export\_interactive\_roc} function, which returns a character
string containing the necessary \proglang{HTML} and
\proglang{JavaScript}. The key interactive feature of these plots is
that the cutoff values nearest to the mouse cursor are displayed when
hovering over the figure. Clicking makes the cutoff label stick until
the next click. The character string can be copy-pasted into an html
document, or better yet, using \pkg{knitr} \citep{knitr}, we can
\texttt{cat} the results and use the option
\texttt{results = \textquotesingle{}asis\textquotesingle{}} so that the
interactive plot is displayed correctly. For examples of interactive
plots and how to incorporate them into \texttt{knitr} documents, see the
package vignette (\texttt{vignette("examples", package = "plotROC")}) or
the webpage https://sachsmc.github.io/plotROC/. Of note is the
\texttt{prefix} option, which allows the user to assign each plot a
unique identifier. This is necessary to prevent conflicts with multiple
plots in a single webpage.

\begin{verbatim}
cat(
  export_interactive_roc(myrocplot, cutoffs = rocdata$c, 
                         font.size = "12px", prefix = "a")
  )
\end{verbatim}

The same \texttt{ggroc} object that we called \texttt{myrocplot} can be
used to generate an ROC plot suitable for use in print. It annotates the
cutoff values and is completely in black and white. A simple example
with the default options is shown in figure \ref{first}.

\begin{verbatim}
plot_journal_roc(myrocplot, rocdata)
\end{verbatim}

\begin{figure}[htbp]
\centering
\includegraphics{figure/print-1.pdf}
\caption{Illustration of ROC curve plot generated by \pkg{plotROC} for
use in print. \label{first}}
\end{figure}

\subsection{Advanced options}\label{advanced-options}

\subsubsection{Click to view confidence
region}\label{click-to-view-confidence-region}

We use the \texttt{ci = TRUE} option in \texttt{calculcate\_roc} and
\texttt{ggroc} to compute confidence regions for points on the ROC curve
using the \citet{clopper1934use} exact method. The significance level
can be specified using the \texttt{alpha} option.

\begin{verbatim}
rocdata <- calculate_roc(M.ex, D.ex, ci = TRUE, alpha = 0.05)
myrocplot <- ggroc(rocdata, label = "Example", ci = TRUE)
\end{verbatim}

For interactive plots, the confidence regions are automatically
detected. When the user clicks on the ROC curve, the confidence region
for the TPF and FPF is overlaid using a grey rectangle. The label and
region stick until the next click.

\begin{verbatim}
cat(
  export_interactive_roc(myrocplot, cutoffs = rocdata$c, 
                         font.size = "12px", prefix = "aci")
  )
\end{verbatim}

For use in print, we pass a small vector of cutoff locations at which to
display the confidence regions. This is shown in figure \ref{conf}.

\begin{verbatim}
plot_journal_roc(myrocplot, rocdata, n.cuts = 10, 
                 ci.at = c(-.5, .5, 2.1))
\end{verbatim}

\begin{figure}[htbp]
\centering
\includegraphics{figure/printci-1.pdf}
\caption{Illustration of \pkg{plotROC} plot with exact confidence
regions. \label{conf}}
\end{figure}

\subsubsection{Multiple ROC curves}\label{multiple-roc-curves}

If you have multiple tests of different types on the same subjects, you
can use the \texttt{calculate\_multi\_roc} function to compute the
empirical ROC curve for each test. Then the \texttt{multi\_ggroc}
function creates the appropriate type of \texttt{ggplot} object.
Confidence regions are not supported for multiple curves at the time of
writing.

\begin{verbatim}
D.ex <- rbinom(100, 1, .5)

fakedata <- data.frame(M1 = rnorm(100, mean = D.ex), 
                       M2 = rnorm(100, mean = D.ex, sd = .4), 
                       M3 = runif(100), D = D.ex)

datalist <- calculate_multi_roc(fakedata, c("M1", "M2", "M3"), "D")
rocplot <- multi_ggroc(datalist)
\end{verbatim}

This multi ggroc object can be passed to the \texttt{plot\_journal\_roc}
and the \texttt{export\_interactive\_roc} functions, as desired.

Labels can be added easily with the \texttt{label} option. The length of
the label element should match the number of plotted curves. The
resulting plot is shown in figure \ref{multi}.

\begin{verbatim}
rocplot <- multi_ggroc(datalist, label = c("M1", "M2", "M3"))
plot_journal_roc(rocplot, datalist)
\end{verbatim}

\begin{figure}[htbp]
\centering
\includegraphics{figure/multi2-1.pdf}
\caption{Illustration of \pkg{plotROC} plot with multiple curves.
\label{multi}}
\end{figure}

\subsubsection{Themes and annotations}\label{themes-and-annotations}

\texttt{plotROC} uses the \texttt{ggplot2} package to create ggplot
objects. Therefore, themes and annotations can be added to
\texttt{ggroc} objects in the usual \texttt{ggplot} way. A
\texttt{plotROC} figure with a new theme, title, axis label, and AUC
annotation is shown in figure \ref{annotate}.

\begin{verbatim}
library(ggplot2)
plot_journal_roc(myrocplot, rocdata) + 
  theme_grey() + 
  geom_abline(intercept = 0, slope = 1, color = "white") + 
  ggtitle("Themes and annotations") + 
  annotate("text", x = .75, y = .25, 
           label = "AUC = 0.80") + 
  ylab("Sensitivity")
\end{verbatim}

\begin{figure}[htbp]
\centering
\includegraphics{figure/print2-1.pdf}
\caption{Using \pkg{ggplot2} themes and annotations with \pkg{plotROC}
objects. \label{annotate}}
\end{figure}

\subsubsection{Other estimation methods}\label{other-estimation-methods}

By default \texttt{calculate\_roc} computes the empirical ROC curve.
There are other estimation methods out there, as we have summarized in
the introduction. Any estimation method can be used, as long as the
cutoff, the TPF and the FPF are returned. Then you can simply pass those
values in a data frame to the \texttt{ggroc} function. For example, let
us use the binormal method to create a smooth curve. This approach
assumes that the test distribution is normal conditional on disease
status.

\begin{verbatim}
D.ex <- rbinom(100, 1, .5)
M.ex <- rnorm(100, mean = D.ex, sd = .5)

mu1 <- mean(M.ex[D.ex == 1])
mu0 <- mean(M.ex[D.ex == 0])
s1 <- sd(M.ex[D.ex == 1])
s0 <- sd(M.ex[D.ex == 0])
c.ex <- seq(min(M.ex), max(M.ex), length.out = 300)

binorm_rocdata <- data.frame(c = c.ex, 
                             FPF = pnorm((mu0 - c.ex)/s0), 
                             TPF = pnorm((mu1 - c.ex)/s1)
                             )
\end{verbatim}

Then we can pass this data.frame to the \texttt{ggroc} function as
before. The example is shown in figure \ref{binorm}.

\begin{verbatim}
binorm_plot <- ggroc(binorm_rocdata, label = "Binormal")
plot_journal_roc(binorm_plot, binorm_rocdata)
\end{verbatim}

\begin{figure}[htbp]
\centering
\includegraphics{figure/binormal-1.pdf}
\caption{Illustration of smooth binormal ROC curve. \label{binorm}}
\end{figure}

\section{How it Works}\label{how-it-works}

\section{Discussion}\label{discussion}


%\bibliographystyle{jss}
\bibliography{plotroc}


\end{document}
